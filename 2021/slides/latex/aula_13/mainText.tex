\documentclass[compress,mathserif]{beamer}
\usetheme{sthlm}

%-=-=-=-=-=-=-=-=-=-=-=-=-=-=-=-=-=-=-=-=-=-=-=-=
%        LOADING BEAMER PACKAGES
%-=-=-=-=-=-=-=-=-=-=-=-=-=-=-=-=-=-=-=-=-=-=-=-=

\usepackage{
booktabs,
datetime,
dtk-logos,
graphicx,
multicol,
pgfplots,
ragged2e,
tabularx,
tikz,
wasysym,
multirow,
float,
caption,
subcaption,
amsmath,
mathptmx,
animate
}

\usepackage[scaled=0.9]{helvet}
\usepackage{courier}

\usefonttheme[onlymath]{serif}

\definecolor{mygreen}{RGB}{113, 166, 70}
\definecolor{myblue}{RGB}{68, 140, 185}
\definecolor{myred}{RGB}{217, 98, 55}
\definecolor{mypurple}{RGB}{83, 65, 126}
\definecolor{solviaveis}{RGB}{188, 207, 241}

\pgfplotsset{compat=1.8}

\usepackage[utf8]{inputenc}
\usepackage[portuguese]{babel}
\usepackage[T1]{fontenc}
\usepackage{newpxtext,newpxmath}
\usepackage{listings}

\lstset{ %
language=[LaTeX]TeX,
basicstyle=\normalsize\ttfamily,
keywordstyle=,
numbers=left,
numberstyle=\tiny\ttfamily,
stepnumber=1,
showspaces=false,
showstringspaces=false,
showtabs=false,
breaklines=true,
frame=tb,
framerule=0.5pt,
tabsize=4,
framexleftmargin=0.5em,
framexrightmargin=0.5em,
xleftmargin=0.5em,
xrightmargin=0.5em
}



%-=-=-=-=-=-=-=-=-=-=-=-=-=-=-=-=-=-=-=-=-=-=-=-=
%        LOADING TIKZ LIBRARIES
%-=-=-=-=-=-=-=-=-=-=-=-=-=-=-=-=-=-=-=-=-=-=-=-=

\usetikzlibrary{
backgrounds,
mindmap
}

%-=-=-=-=-=-=-=-=-=-=-=-=-=-=-=-=-=-=-=-=-=-=-=-=
%        BEAMER OPTIONS
%-=-=-=-=-=-=-=-=-=-=-=-=-=-=-=-=-=-=-=-=-=-=-=-=

\setbeameroption{show notes}

%-=-=-=-=-=-=-=-=-=-=-=-=-=-=-=-=-=-=-=-=-=-=-=-=
%        BEAMER COMMANDS
%-=-=-=-=-=-=-=-=-=-=-=-=-=-=-=-=-=-=-=-=-=-=-=-=


%-=-=-=-=-=-=-=-=-=-=-=-=-=-=-=-=-=-=-=-=-=-=-=-=
%
%	PRESENTATION INFORMATION
%
%-=-=-=-=-=-=-=-=-=-=-=-=-=-=-=-=-=-=-=-=-=-=-=-=

\title{Dualidade em Programação Linear}
\subtitle{DCE692 - Pesquisa Operacional}
%\date{\small{\jobname}}
\author{\texttt{Iago Carvalho}}
\institute{\texttt{Departamento de Ciência da Computação}}

\hypersetup{
pdfauthor = {Iago A. Carvalho},      
pdfsubject = {Pesquisa Operacional},
pdfkeywords = {},  
pdfmoddate= {D:\pdfdate},          
pdfcreator = {WriteLaTeX}
}

\begin{document}

\begin{frame}
\titlepage

\end{frame}

%% --------------------------------------------------------

\begin{frame}{Problemas duais}

Problemas de programação linear são descritos utilizando um conjunto de equações lineares
\begin{itemize}
    \item Função objetivo
    \item Variáveis
    \item Restrições
\end{itemize}

Estes são chamados de problemas \textit{primais}

\vspace{0.5cm}

Todo problema de programação linear possui um problema \textit{dual} associado
\begin{itemize}
    \item Problema de programação linear
    \item Cada variável do primal torna-se uma restrição no dual
    \item Cada restrição do primal torna-se uma variável no dual
    \item Sentido da função objetivo é invertida
\end{itemize}
\end{frame}

%% --------------------------------------------------------

\begin{frame}{Problemas duais}

\begin{columns}[T]
    \begin{column}{.44\textwidth}
        \centering Primal
        $$\min~~c\,x$$
        $$A\,x \leq b$$
        $$x \geq 0$$
    \end{column}
    \begin{column}{.49\textwidth}
        \centering Dual
        $$ \max~~z\,b $$
            $$ z\,A \leq c$$
            $$ z \geq 0 $$
    \end{column}
\end{columns}
\end{frame}

%% --------------------------------------------------------

\begin{frame}{Teorema fraco da dualidade}

Se $x$ for uma solução viável para o problema primal e \textit{z} for uma solução viável para o problema dual, então

$$
c\,x \leq z\,b
$$
\end{frame}

%% --------------------------------------------------------

\begin{frame}{Teorema forte da dualidade}

Se $x^*$ a solução ótima para o problema primal e $z^*$ uma solução ótima para o problema dual, então

$$
c\,x^* = z^*\,b
$$

\end{frame}

%% --------------------------------------------------------

\begin{frame}{Teorema da dualidade}

Se o problema primal tiver soluções viáveis e uma função objetivo limitada
\begin{itemize}
    \item O mesmo acontece no dual
\end{itemize}

\vspace{0.5cm}

Se o problema primal tiver soluções viáveis e uma função objetivo ilimitada
\begin{itemize}
    \item Dual não terá soluções viáveis
\end{itemize}

\vspace{0.5cm}

Se o problema primal não tiver nenhuma solução viável
\begin{itemize}
    \item Dual não terá nenhuma solução viável ou uma função objetivo ilimitada
\end{itemize}
\end{frame}

\end{document}
