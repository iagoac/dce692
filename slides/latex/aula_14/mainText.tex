\documentclass[compress,mathserif]{beamer}
\usetheme{sthlm}

%-=-=-=-=-=-=-=-=-=-=-=-=-=-=-=-=-=-=-=-=-=-=-=-=
%        LOADING BEAMER PACKAGES
%-=-=-=-=-=-=-=-=-=-=-=-=-=-=-=-=-=-=-=-=-=-=-=-=

\usepackage{
booktabs,
datetime,
dtk-logos,
graphicx,
multicol,
pgfplots,
ragged2e,
tabularx,
tikz,
wasysym,
multirow,
float,
caption,
subcaption,
amsmath,
mathptmx,
animate
}

\usepackage[scaled=0.9]{helvet}
\usepackage{courier}

\usefonttheme[onlymath]{serif}

\definecolor{mygreen}{RGB}{113, 166, 70}
\definecolor{myblue}{RGB}{68, 140, 185}
\definecolor{myred}{RGB}{217, 98, 55}
\definecolor{mypurple}{RGB}{83, 65, 126}
\definecolor{solviaveis}{RGB}{188, 207, 241}

\pgfplotsset{compat=1.8}

\usepackage[utf8]{inputenc}
\usepackage[portuguese]{babel}
\usepackage[T1]{fontenc}
\usepackage{newpxtext,newpxmath}
\usepackage{listings}

\lstset{ %
language=[LaTeX]TeX,
basicstyle=\normalsize\ttfamily,
keywordstyle=,
numbers=left,
numberstyle=\tiny\ttfamily,
stepnumber=1,
showspaces=false,
showstringspaces=false,
showtabs=false,
breaklines=true,
frame=tb,
framerule=0.5pt,
tabsize=4,
framexleftmargin=0.5em,
framexrightmargin=0.5em,
xleftmargin=0.5em,
xrightmargin=0.5em
}



%-=-=-=-=-=-=-=-=-=-=-=-=-=-=-=-=-=-=-=-=-=-=-=-=
%        LOADING TIKZ LIBRARIES
%-=-=-=-=-=-=-=-=-=-=-=-=-=-=-=-=-=-=-=-=-=-=-=-=

\usetikzlibrary{
backgrounds,
mindmap
}

%-=-=-=-=-=-=-=-=-=-=-=-=-=-=-=-=-=-=-=-=-=-=-=-=
%        BEAMER OPTIONS
%-=-=-=-=-=-=-=-=-=-=-=-=-=-=-=-=-=-=-=-=-=-=-=-=

\setbeameroption{show notes}

%-=-=-=-=-=-=-=-=-=-=-=-=-=-=-=-=-=-=-=-=-=-=-=-=
%        BEAMER COMMANDS
%-=-=-=-=-=-=-=-=-=-=-=-=-=-=-=-=-=-=-=-=-=-=-=-=


%-=-=-=-=-=-=-=-=-=-=-=-=-=-=-=-=-=-=-=-=-=-=-=-=
%
%	PRESENTATION INFORMATION
%
%-=-=-=-=-=-=-=-=-=-=-=-=-=-=-=-=-=-=-=-=-=-=-=-=

\title{Dualidade em Programação Linear}
\subtitle{DCE692 - Pesquisa Operacional}
%\date{\small{\jobname}}
\author{\texttt{Iago Carvalho}}
\institute{\texttt{Departamento de Ciência da Computação}}

\hypersetup{
pdfauthor = {Iago A. Carvalho},      
pdfsubject = {Pesquisa Operacional},
pdfkeywords = {},  
pdfmoddate= {D:\pdfdate},          
pdfcreator = {WriteLaTeX}
}

\begin{document}

\begin{frame}
\titlepage

\end{frame}

%% --------------------------------------------------------

\begin{frame}{Problemas duais}

Problemas de programação linear são descritos utilizando um conjunto de equações lineares
\begin{itemize}
    \item Função objetivo
    \item Variáveis
    \item Restrições
\end{itemize}

Estes são chamados de problemas \textit{primais}

\vspace{0.5cm}

Todo problema de programação linear possui um problema \textit{dual} associado
\begin{itemize}
    \item Problema de programação linear
    \item Cada variável do primal torna-se uma restrição no dual
    \item Cada restrição do primal torna-se uma variável no dual
    \item Sentido da função objetivo é invertida
\end{itemize}
\end{frame}

%% --------------------------------------------------------

\begin{frame}{Problemas duais}

\begin{columns}[T]
    \begin{column}{.44\textwidth}
        \centering Primal
        $$\min~~c\,x$$
        $$A\,x \leq b$$
        $$x \geq 0$$
    \end{column}
    \begin{column}{.49\textwidth}
        \centering Dual
        $$ \max~~z\,b $$
            $$ z\,A \geq c$$
            $$ z \geq 0 $$
    \end{column}
\end{columns}
\end{frame}

%% --------------------------------------------------------

\begin{frame}{Contextualização do problema dual}

Uma indústria deseja produzir três tipos de molhos a partir de ketchup e mostarda. 
\begin{itemize}
    \item A indústria possui, ao todo, 80kg de ketchup e 30kg de mostarda
    \item O objetivo é produzir molhos de tal forma que o lucro da venda seja maximizado
\end{itemize}

$$\begin{matrix}
        \max & 10x_1 & + & 7x_2 & + & 15x_3 \\ 
             & 5x_1 & + & 4x_2 & + & x_3 & \leq 80 \\
             & 2x_1 & + & 3x_2 & + & 5x_3& \leq 30 \\
             & & & x_i & & & \geq 0, & \forall i \in \{1, 2, 3\}
        \end{matrix}    
$$

\end{frame}

%% --------------------------------------------------------

\begin{frame}{Contextualização do problema dual}

Suponha que outra empresa deseje adquirir a fábrica de molhos
\begin{itemize}
    \item Esta segunda empresa deseja realizar uma boa compra para poder lucrar
    \item Deseja-se precificar o preço de venda de cada kg de ketchup e mostarda
\end{itemize}

\vspace{0.5cm}

Para que o vendedor não saia no prejuízo, ele também tem que ter algum lucro
\begin{itemize}
    \item Para isto, pode-se utilizar o dual do modelo de programação linear
\end{itemize}

\end{frame}

%% --------------------------------------------------------

\begin{frame}{Contextualização do problema dual}

$$\begin{matrix}
        \max & \textcolor{red}{10}x_1 & + & 7x_2 & + & 15x_3 \\ 
             & \textcolor{sthlmLightBlue}{5}x_1 & + & 4x_2 & + & x_3 & \leq 80 \\
             & \textcolor{sthlmLightBlue}{2}x_1 & + & 3x_2 & + & 5x_3& \leq 30 \\
             & & & x_i & & & \geq 0, & \forall i \in \{1, 2, 3\}
        \end{matrix}    
$$

\vspace{0.5cm}

Uma unidade do molho $x_1$ é vendido por 10. Além disso, consome
\begin{itemize}
    \item 5kg de ketchup
    \item 2kg de mostarda
\end{itemize}

\vspace{0.5cm}

Deste modo, só vale a pena vender 5kg de ketchup e 2kg de mostarda por 10 ou mais unidades monetárias, isto é

$$
5z_1 + 2z_2 \geq 10
$$
\end{frame}

%% --------------------------------------------------------

\begin{frame}{Contextualização do problema dual}

$$\begin{matrix}
        \max & 10x_1 & + & 7x_2 & + & 15x_3 \\ 
             & 5x_1 & + & 4x_2 & + & x_3 & \leq 80 \\
             & 2x_1 & + & 3x_2 & + & 5x_3& \leq 30 \\
             & & & x_i & & & \geq 0, & \forall i \in \{1, 2, 3\}
        \end{matrix}    
$$

\vspace{0.5cm}

Da mesma forma para os outros dois molhos. Ao fim, temos que

$$\left.\begin{matrix}
         5\,z_1 & + & 2\,z_2 & \geq & 10 \\
         4\,z_1 & + & 3\,z_2 & \geq & 7 \\
         z_1 & + & 5\,z_2 & \geq & 15
        \end{matrix}\right\} \leftarrow \textnormal{vantagem do vendedor}
$$
\end{frame}

%% --------------------------------------------------------

\begin{frame}{Contextualização do problema dual}

Já o comprador quer minimizar seus gastos ao adquirir a empresa de molhos. 

Seja $z_1$ o valor que ele desembolsará para comprar cada kg de ketchup e $z_2$ por cada kg de mostarda. 

Então, temos que o comprador deseja

$$
    \min~~80\,z_1 + 30\,z_2
$$

\end{frame}

%% --------------------------------------------------------

\begin{frame}{Contextualização do problema dual}

Ao fim, temos que o comprador deseja otimizar um problema de otimização linear
\begin{itemize}
    \item Este problema é o dual associado ao problema de produção primal
\end{itemize}

$$\begin{matrix}
        \min & 80\,z_1 & + & 30\,z_2 \\
         & 5\,z_1 & + & 2\,z_2 & \geq & 10 \\
         & 4\,z_1 & + & 3\,z_2 & \geq & 7 \\
         & z_1 & + & 5\,z_2 & \geq & 15 \\
         & & z_i & & \geq & 0, & \forall i \in \{1, 2\}
        \end{matrix}
$$

\end{frame}

%% --------------------------------------------------------

\begin{frame}{Contextualização do problema dual}

\centering Primal
        $$\begin{matrix}
        \max & 10x_1 & + & 7x_2 & + & 15x_3 \\ 
             & 5x_1 & + & 4x_2 & + & x_3 & \leq 80 \\
             & 2x_1 & + & 3x_2 & + & 5x_3& \leq 30 \\
             & & & x_i & & & \geq 0, & \forall i \in \{1, 2, 3\}
        \end{matrix}    
$$

\vspace{0.7cm}

        \centering Dual
        $$\begin{matrix}
        \min & 80\,z_1 & + & 30\,z_2 \\
         & 5\,z_1 & + & 2\,z_2 & \geq & 10 \\
         & 4\,z_1 & + & 3\,z_2 & \geq & 7 \\
         & z_1 & + & 5\,z_2 & \geq & 15 \\
         & & z_i & & \geq & 0, & \forall i \in \{1, 2\}
        \end{matrix}
$$
\end{frame}

%% --------------------------------------------------------

\begin{frame}{Primal-dual: forma geral}

\centering Primal
$$
\begin{matrix} 
\max & c_1 x_1 & + & c_2 x_2 & + & \dots & + & c_n x_n & \\
     & a_{11} x_1 & + & a_{12} x_2 & + & \dots  & + & a_{1n} x_n & \leq & b_1 \\
     & a_{21} x_1 & + & a_{22} x_2 & + & \dots  & + & a_{2n} x_n & \leq & b_2 \\
     & \vdots     &   & \vdots     &   & \ddots &   & \vdots     &           & \vdots \\
     & a_{m1} x_1 & + & a_{m2} x_2 & + & \dots  & + & a_{mn} x_n & \leq & b_m \\
     & x_1,       &   & x_2,       &   & \dots  &   &        x_n & \geq & 0 \\
\end{matrix}
$$

\vspace{0.3cm}

        \centering Dual
        $$\begin{matrix} 
\min& b_1\,z_1 & + & b_2\,z_2 & + & \dots & + & b_m\,z_m & \\
     & a_{11}\,z_1 & + & a_{12}\,x_2 & + & \dots  & + & a_{1n}\,z_m & \geq & c_1 \\
     & a_{21}\,z_1 & + & a_{22}\,x_2 & + & \dots  & + & a_{2n}\,z_m & \geq & c_2 \\
     & \vdots     &   & \vdots     &   & \ddots &   & \vdots     &           & \vdots \\
     & a_{n1}\,z_1 & + & a_{n2}\,z_2 & + & \dots  & + & a_{nm}\,z_m & \geq & c_n \\
     & z_1,       &   & z_2,       &   & \dots  &   &        z_m & \geq & 0 \\
\end{matrix}
$$
\end{frame}

\end{document}